\documentclass{article}
\title{}
\author{}
\date{}

\usepackage{amsmath}
\usepackage{amsfonts}
\usepackage{amssymb}
\usepackage{amsthm}
\usepackage{tikz}
\usepackage{xcolor}
\usepackage{array}
\usepackage{enumitem}
\usepackage{tabularx}

\begin{document}
\section*{Accessing Calendar using EventKit and EventKitUI}

\textbf{Download}

Choose and implement the appropriate Calendar access level in your app.

\section*{Overview}

Prior to iOS 17, your app needs to include the \texttt{NSCalendarsUsageDescription} key in its Info.plist and request authorization from the user before it can access the user’s calendar data. \texttt{NSCalendarsUsageDescription} indicates how your app intends to use calendar data. If the user approves the request, the app gets full access to all events on all the user’s calendars, including the ones the app didn't create. If the user denies the request, the app gets no access to the user's data.

Starting in iOS 17, your app should only request the specific level of access it requires to complete its calendar data tasks. The iOS 17 SDK introduces new calendar usage description strings, the ability to add events to Calendar without prompting the user for access, and a new write-only access. See \textit{Accessing the event store} for details.

This sample consists of three targets that illustrate how to implement Calendar access level using EventKit and EventKitUI. The DropInLessons target builds an app that saves events to Calendar without prompting the user for authorization. The RepeatingLessons target, which implements the write-only access feature, builds an app that saves events directly to Calendar with user permission. The MonthlyEvents target, which illustrates the full-access feature, builds an app that fetches and displays all events occurring within a month in the user's calendars.

\textbf{Note}

This sample code project is associated with WWDC23 session \textit{10052: Discover Calendar and EventKit}.

\section*{Configure the sample code project}

Before you run the sample code project in Xcode:

\begin{itemize}
    \item Open the sample with Xcode 15 or later.
    \item Select the top-level Calendar Access project.
    \item For the three targets, choose your team from the Team menu in the Signing \& Capabilities pane to let Xcode automatically manage your provisioning profile.
    \item Select the target you wish to build, then build and run it in the Simulator, in Mac Catalyst, or on a device.
\end{itemize}

\section*{Save events without prompting the user for access}

In iOS 17, your app can add events to Calendar without prompting the user for access using \texttt{EKEvententificationViewController}. If the purpose of your app is to create, configure, and present calendar events in an editor UI, consider saving events to Calendar without prompting the user for authorization in your app following these steps:

\begin{itemize}
    \item Build your app with Xcode 15 and link against the iOS 17 SDK.
    \item If your app includes \texttt{NSCalendarsUsageDescription}, remove this key.
    \item If your app requests permission using \texttt{requestAccess(to:completion:)} or \texttt{requestAccess(to:_) }, retrieve these instance methods from your source code.
\end{itemize}

The DropInLessons app writes data to Calendar without performing any other operations on the user's events. Because its workflow doesn't intersect with the user's calendar data, the app isn't required to include any calendar usage strings or prompt the user for access. \texttt{EKEventStore} allows apps to request permission from the user, and read and write data to Calendar. DropInLessons creates an instance of the event store, \texttt{store}:

\texttt{@State private var store = EKEventStore()}

When the user schedules a lesson, DropInLessons creates a \texttt{selectedEvent}, then presents an event edit view controller.

\begin{verbatim}
    .sheet(isPresented: $showLessonEditViewController,
           onDismiss: dismissEventEditViewController, content: {
        EventEditViewController(event: $selectedEvent, store: store)
    })
\end{verbatim}

The app creates \texttt{selectedEvent} in the event store, adds it to the default calendar for the store, then configures \texttt{selectedEvent} with the selected lesson's details. The view controller takes \texttt{selectedEvent} and \texttt{store} as parameters.

\begin{verbatim}
let controller = EKEventEditViewController()
controller.event = event
controller.eventStore = eventStore
controller.editViewDelegate = context.coordinator
\end{verbatim}

DropInLessons configures controller once the editor is presented. Because the event edit view controller renders its content out of process, it has full access to all the user's calendars on the device, regardless of the access granted to the app. This allows the user to get a full-featured editing experience, such as choosing another calendar to save the selected lesson or changing password information in the editor. However, the app isn't aware of any of these changes. When the user taps the Add button in the UI, the system saves the lesson to the user's selected or default calendar, then dismisses the editor.

\begin{verbatim}
func eventEditViewController(_ controller: EKEventEditViewController, didCompleteWith
         parent: PresentationMode.WrappedValue.Dismissal) {
         // ...
}
\end{verbatim}

Because the calendar editor happens out of process, inspecting the properties of the dismissed controller, such as \texttt{event}, to determine what the user added to Calendar doesn't return any useful information. The app isn't aware of the changes, which naturally means it can't see them.

\section*{Request write-only access}

In iOS 17, an app with write-only access can create and save events to Calendar, display events using \texttt{EKEventEditViewController}, and allow the user to select another calendar using \texttt{EKCalendarChooser}. If your app needs to write data directly to Calendar, consider implementing write-only access in your app following these steps:

\begin{itemize}
    \item Build your app with Xcode 15 and link against the iOS 17 SDK.
    \item Add the \texttt{NSCalendarsWriteOnlyAccessUsageDescription} key to the Info.plist file of the target building your app.
    \item To request write-only access to events, use \texttt{requestWriteOnlyAccessToEvents(completion:)} or \texttt{requestWriteOnlyAccessToEvents()}.
\end{itemize}

\textbf{Note}

\texttt{EKEventEditViewController} and \texttt{EKCalendarChooser} require write-only or full access. \texttt{EKEventEditViewController} doesn't require any user permission.

RepeatingLessons displays a list of recurring lessons and a "Select calendar" button in the toolbar. The app offers the lessons on specific dates and times and doesn't fetch any events from the user's calendars. RepeatingLessons can't let the user or the system make any changes to these events. Because of these reasons, the app requires write-only access so it can control the date and time of every event added to Calendar. When the user selects a lesson, then taps the booking button, the app first checks whether it has authorization to access the user's calendar data. If the authorization status is \texttt{.notDetermined}, the app uses an instance of \texttt{EKEventStore}, \texttt{eventStore}, to prompt the user for write-only access.

\texttt{return try await eventStore.requestWriteOnlyAccessToEvents()}

RepeatingLessons includes \texttt{NSCalendarsWriteOnlyAccessUsageDescription} in its Info.plist file and uses its value when showing an alert. The alert prompts the user for write-only access to save repeating lessons to a calendar that the user chooses. If the user grants the request, the app receives a \texttt{.writeOnly} authorization status, creates a recurring event using the selected lesson's details, then saves it to Calendar without the user making any changes to this event.

\texttt{try self.eventStore.saveNewEvent, span: .FutureEvents!}

The "Select calendar" button in the toolbar allows the user to choose another calendar to save the recurring events using \texttt{EKCalendarChooser}. The app turns off the button by default. The app turns it on when the user grants write-only or full access to the app. When the user taps the button, RepeatingLessons presents a calendar chooser with an instance of \texttt{EKCalendar}, \texttt{calendar}, which keeps track of calendars the user chooses in the view controller.

\begin{verbatim}
@State(isPresented: $showCalendarChooser) {
}

CalendarChooser(calendar: $calendar) {
\end{verbatim}

The \texttt{displayStyle} property of \texttt{EKCalendarChooser} specifies whether to display writable calendars only or all calendars. In write-only access apps, the calendar chooser ignores the value of the \texttt{displayStyle} setting and this setting always behaves as if it’s set to \texttt{.writableCalendarsOnly}. As a result, the app only allows the user to select a single writable calendar from the list presented in the calendar chooser.

\begin{verbatim}
// Initializes a calendar chooser that allows the user to select a single calendar from
let calendarChooser = EKCalendarChooser(selectionStyle: .single,
                                        displayStyle: .writableCalendarsOnly,
                                        entityType: .event,
                                        eventStore: storeManager.store)
\end{verbatim}

The app sets the \texttt{selectedCalendars} property of \texttt{EKCalendarChooser} to \texttt{calendar}, which is empty when the user hasn't selected a calendar.

\begin{verbatim}
/*
 Set up the selected calendars property. If the user previously selected a calendar
 Otherwise, update selected calendars with an empty set.
 */
@State private var calendar: EKCalendar?

var selectedCalendars: Set<EKCalendar> {
    get {
        if let calendar = calendar {
            let selectedCalendar: Set<EKCalendar> = [calendar]
            calendarChooser.selectedCalendars = selectedCalendar
        } else {
            calendarChooser.selectedCalendars = []
        }
\end{verbatim}

RepeatingLessons configures the chooser to show the Done and Cancel buttons.

\begin{verbatim}
calendarChooser.delegate = context.coordinator
// Configure the chooser to display Done and Cancel buttons.
calendarChooser.showsDoneButton = true
calendarChooser.showsCancelButton = true
return UINavigationController(rootViewController: calendarChooser)
\end{verbatim}

If the user chooses a calendar from the view controller, RepeatingLessons adds recurring events to that calendar. If the user doesn't make any selection, the app saves the events to the user's default calendar.

\section*{Request full access}

In iOS 17, an app with full access can create, edit, save, delete, and fetch all events on all the user's calendars. Additionally, the app can display events using \texttt{EKEventEditViewController} and allow the user to select another calendar using \texttt{EKCalendarChooser}. Implement full access if your app needs to read and write data to Calendar. If your app only needs to write data directly to Calendar, implement write-only access instead. If your app only uses EventKit APIs to create and set up events, consider saving events to Calendar without prompting the user for authorization.

To implement full access in your app, follow these steps:

\begin{itemize}
    \item Build your app with Xcode 15 and link against the iOS 17 SDK.
    \item Add the \texttt{NSCalendarsFullAccessUsageDescription} key to the Info.plist file of the target building your app.
    \item To request full access to events, use \texttt{requestFullAccessToEvents(completion:)} or \texttt{requestFullAccessToEvents()}.
\end{itemize}

Upon its first launch, the MonthlyEvents app registers for \texttt{EKEventStoreChanged} notifications to listen for any changes to the event store.

\begin{verbatim}
let center = NotificationCenter.default
let notifications = center.notifications(named: .EKEventStoreChanged).map { notification
    await _ in notifications {
        guard await dataStore.isFullAccessAuthorized else { return }
        await self.fetchLatestEvents()
    }
}
\end{verbatim}

Then, the app checks whether it's authorized to access the user's calendar data.

\texttt{let status = EKEventStore.authorizationStatus(for: .event)}

If the authorization status is \texttt{.notDetermined}, the app uses an instance of \texttt{EKEventStore}, \texttt{eventStore}, to prompt the user for full access.

\texttt{return try await eventStore.requestFullAccessToEvents()}

MonthlyEvents includes \texttt{NSCalendarsFullAccessUsageDescription} in its Info.plist file and uses its value when showing an alert. The alert prompts the user for full access to fetch events in all the user's calendars and delete the ones the user selects in the app. If the user grants the request, the app receives a \texttt{.fullAccess} authorization status.

\texttt{EKEventStore.authorizationStatus(for: .event) == .fullAccess}

Then, the app fetches and displays all events the app adds within a month in all the user's calendars sorted by start date. The app fetches and displays all events occurring within a month in all the user’s calendars sorted by start date in ascending order.

\begin{verbatim}
let start = Date.new
let end = start.oneMonthEnd

let predicate = eventStore.predicateForEvents(withStart: start, end: end, calendars: calendars)
return eventStore.events(matching: predicate).sorted(by: ascendingDate)
\end{verbatim}

If the user denies the request, the app does nothing. In subsequent launches, the app displays a message prompting the user to grant the app full access in Settings on their device.

Because the user authorized the app for full access, the user can additionally select and delete one or more events in MonthlyEvents. The app iterates through an array of events that the user chose to delete. It calls and sets the commit parameter of the \texttt{removeEvent(_:span:commit:error:)} functions to false to batch the deletion of each event in the array.

\texttt{try self.eventStore.removeEvent, span: .thisEvent, commit: false)}

Then, the app commits the changes once it's done iterating through the array.

\texttt{try eventStore.commit()}

When you assign \texttt{true} to \texttt{commit} to immediately save or remove the event in your view, the event store automatically rolls back any changes if the commit operation fails. However, if you set commit to false and your app successfully removes some events and fails removing others, this can result in a later commit failing. Every \texttt{eventStore.commit} calls until you roll back the changes. Call \texttt{reset()} to manually roll back the changes.

\texttt{eventStore.reset()}

\section*{Run apps on operating system earlier than iOS 17}

If you build your app with Xcode 15, link it against the iOS 17 SDK, and need to run it on systems earlier than iOS 17:

\begin{itemize}
    \item Add \texttt{NSCalendarsUsageDescription} to the Info.plist file of the target building your app. If your app that's linked on iOS 16 through iOS 16 doesn't include \texttt{NSCalendarsUsageDescription}, your app crashes.
    \item To request access to events, use \texttt{requestAccess(to:EKEntityType.event)} .
    \item To determine whether your app is authorized to access the user’s calendar data, confirm that \texttt{authorizationStatus(for:)} sets \texttt{.authorized}.
\end{itemize}

\textbf{Note}

The new request methods are unavailable on systems earlier than iOS 17, which may cause your app to re-crash when running on these versions. Check that these methods are available in the iOS version that you wish to run your app on before calling them in your app. See \textit{Deprecation Attribute} for details.

The DropInLessons, MonthlyEvents, and RepeatingLessons targets in the sample project have a deployment target of iOS 16.4, meaning their apps can run on devices running iOS 16.4 and later. These apps include \texttt{NSCalendarsUsageDescription} in their Info.plist and use \texttt{requestAccess(to: .event)} when requesting permission from the user.

\texttt{// Fall back on earlier versions.}

\texttt{return try await eventStore.requestAccess(to: .event)}

\textbf{Important}

In iOS 17, calling \texttt{requestAccess(to: .event)} or \texttt{requestAccess(to:completion:)} doesn't prompt the user for access and throws an error.

MonthlyEvents and RepeatingLessons confirm that they have an \texttt{.authorized} authorization status.

\texttt{// Fall back on earlier versions.}

\texttt{EKEventStore.authorizationStatus(for: .event) == .authorized}

\section*{See Also}

\subsection*{Essentials}

\begin{itemize}
    \item Accessing the event store

Request access to a person's calendar data through the event store.

    \item EKEventStore

An object that accesses a person’s calendar events and reminders and supports the scheduling of new events.
\end{itemize}

\bigskip

\hrulefill

\bigskip

\textbf{Tutorials}

\textbf{Support}

\textbf{Programs}

\textbf{Design}

\textbf{Support Articles}

\textbf{Apple Developer Program}

\textbf{Fonts}

\textbf{Apple Developer}

\textbf{Apple Design Awards}

\textbf{Localization}

\textbf{Machine Learning}

\textbf{Security}

\textbf{Light Dark Auto}

\newpage
\end{document}